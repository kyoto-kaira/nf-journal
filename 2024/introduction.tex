2023年は生成AIが世間一般に広く認知された年となり、今年の新語・流行語大賞には「生成AI」「チャットGPT」がノミネートされました。
さらに「観る将」という言葉も新語・流行語大賞にノミネートされました。
最近では将棋のライブ配信でAIの評価値が表示されることが多くなり、将棋の知識がなくてもどちらが勝勢なのかが容易に分かるようになったことが「観る将」の流行につながりました。

\vskip\baselineskip

このようにさまざまなAIに一般の人々が触れる機会が増え、AIがより身近な存在になったと言えるでしょう。
本会誌では、今年注目を集めたChatGPT(第I\hspace{-1.2pt}I部)、将棋AI(第I\hspace{-1.2pt}I\hspace{-1.2pt}I部)、画像生成AI(第I\hspace{-1.2pt}V部)についての記事を掲載しています。

\vskip\baselineskip

第I部では、AIの歴史と未来について紹介しています。
もしかすると皆さんの中には、AIはつい最近登場したものだと思っている方がいるかもしれませんが、
実はAIは1950年代から研究されており、幾度かのAIブームを経て今に至っています。

\vskip\baselineskip

第I\hspace{-1.2pt}I\hspace{-1.2pt}I部では、私が独自に開発した"引き分けを目指す"オセロAIについての解説を行っています。
NFの展示会場でも体験できるようになっていますので、ぜひ遊んでみてください。

\vskip\baselineskip

第V部では、AIの判断理由を説明する手法であるXAIについて紹介しています。
実はAIは「なぜその予測をしたのか」を説明することが苦手です。
研究者たちは何とかその問題を解決しようと様々な解釈手法を考案しており、中でも代表的な3つの手法を紹介しています。

\vskip\baselineskip

本会誌は、KaiRAの会員たちが協力して作成してくれました。
KaiRAでは毎週木曜日に勉強会を開催しており、活発な議論が行われています。
本会誌を読んでAIに興味を持っていただけたら、ぜひあなたもKaiRAで私たちと一緒にAIを学んでみませんか。

\vskip\baselineskip
\rightline{京都大学人工知能研究会KaiRA\ 会長}
\rightline{理学部\ (地球物理学)\ 3回生}
\rightline{松田拓巳}
