本誌を手に取っていただき、ありがとうございます。
KaiRA会長の松田です。

\vskip\baselineskip

2022年11月のChatGPT公開以降、生成AIは驚異的なスピードで進化を遂げ、その普及は社会のあらゆる分野に広がっています。
現在では、文章だけでなく、画像、音声、さらには動画など、多様なデータ形式を扱える生成AIが次々と登場し、AI技術の社会実装がさらに加速しています。

\vskip\baselineskip

このような生成AIの開発には、高度な技術、膨大なデータセット、そして莫大な計算リソースが必要です。
そのため、多くの場合、大手企業や大学などの研究機関が主導する分野とされています。

\vskip\baselineskip

しかし、私たちは京都大学人工知能\textbf{研究会}の一員として、限られたリソースの中でも果敢に生成AIの開発に挑戦しました。
大規模プロジェクトには及ばない規模ではありますが、自ら手を動かして取り組むことで、生成AI技術の核心やその学習過程の困難さを直に体験し、多くのことを学ぶことができると信じています。

\vskip\baselineskip

夏休みから11月祭にかけて、私たちはオリジナルAIの開発プロジェクトを立ち上げました。
メンバー全員でアイデアを出し合い、試行錯誤を重ねながら、開発を進めてきました。
確かに、AIの学習には数多くの壁が立ちはだかり、必ずしも思い描いた成果を得るには至りませんでした。
しかし、それでも得られた経験と知見は非常に価値のあるものでした。

\vskip\baselineskip

本会誌では、私たちが試行錯誤しながら開発したオリジナルAIの挑戦と、その過程で得た教訓をご紹介します。
ぜひご一読いただき、私たちの取り組みをお楽しみいただけますと幸いです。

\vskip\baselineskip
\rightline{京都大学人工知能研究会KaiRA\ 会長}
\rightline{松田拓巳}
\rightline{2024年11月吉日}
