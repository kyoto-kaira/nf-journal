本誌を手に取ってくださり、ありがとうございます。
京都大学人工知能研究会 KaiRA 会長の岡本和優です。

\vskip\baselineskip

人工知能という言葉は、いまや特別な響きをもたなくなりました。
つい数年前——GPT-2がその安全性の懸念から一般公開すらされていなかった頃は、AIは研究室や巨大企業だけが扱う「特別な道具」でした。
それが今では、インターネットさえつながっていれば、誰もがAIを使ったサービスを利用でき、日常の中に自然に溶け込むようになっています。

\vskip\baselineskip

AI を「使う」ことについては確かに民主化が進みましたが、AI を「作る」「研究する」側の世界は、まだ十分に開かれているとは言えません。
巨大企業だけが膨大な計算資源を用いてモデルを訓練し、研究の主導権はますます閉じた領域に集中しています。
かつてオープンであると信じられたAI研究の世界は、気づけばマネーゲームの世界へと収束しつつあります。

だからこそ、私たちのような小さなコミュニティが、自分たちの手で考え、試し、つくってみることに意味があります。
シンギュラリティが来るかどうかといった大きな議論よりも、まずは自分たちの身近な問いから技術を生み出し、
ほんの少しでも未来をよくする試みを続けていくこと——それこそが、AI時代の創造性だと私は思っています。

大規模モデルが世界を塗り替えようとしている今でも、「自分たちで未来を形にしよう」とする学生たちの活動は決して色あせません。
むしろ、この時代だからこそ、その小さな挑戦には大きな価値があるのではないでしょうか。

\vskip\baselineskip

本誌には、学生が主体となって取り組んだ研究や開発の記録が収められています。
そこに書かれているのは、華やかな成功ばかりではありません。しかし、そこで生まれた「試行錯誤の跡」こそが、この会誌の魅力です。

どんな問いを持ち、どう試して、どこでつまずき、何を得たのか。
そのひとつひとつが、これからの技術や研究につながる小さな種だと思っています。

読者のみなさんには、ぜひこの 「挑戦のプロセス」 を楽しんでいただければ幸いです。
巨大資本や巨大モデルが主導する時代だからこそ、
学生が自分の興味や疑問から始めた小さな試みには、ほかにはない輝きがあると信じています。

\vskip\baselineskip
\rightline{京都大学人工知能研究会KaiRA\ 会長}
\rightline{岡本和優}
\rightline{2025年11月吉日}